\documentclass{resume} % Use the custom resume.cls style
\usepackage[left=0.5in,top=0.5in,right=0.5in,bottom=0.5in]{geometry} % Document margins
\usepackage{hyperref}

\name{Eva Marie Biedron} % Your name
\address{\href{mailto:embiedron@fas.harvard.edu}{embiedron@fas.harvard.edu} \\ \href{github.com/Eva-Biedron}{github.com/Eva-Biedron}} 

\begin{document}
			
	\begin{rSection}{Education}
		{\bf Vanderbilt University} \hfill {\em 2018} \\ 
		{\em MS,} Earth \& Environmental Sciences 
		
		{\bf University of Oregon, Clark Honors College}  \hfill {\em 2016} \\
		{\em BS, cum laude,} Geological Sciences, Biology, Chemistry (minor)
	
		{\bf Harvard University, Extension School} \hfill {\em 2021} \\
		{\em Graduate Certificate,} Project Management
	\end{rSection}
	
	\begin{rSection}{Experience}	
		{\bf Museum of Comparative Zoology }{Curatorial Assistant II} \hfill {\em 2018 - present} \\
		{Harvard University} \hfill {Cambridge, MA} 
		\begin{itemize}
			\itemsep -0.5em \vspace{-0.5em} \small
			\item Conserve, track, and facilitate user access to 1000000+ fossils using a custom institutional database.
			\item Manage projects \& direct team-members to a) process 1500+ incoming specimens, b) locate materials \& update database, and c) transcribe Spanish-language documents.
			\item Aid in annual collection usage statistic preparation \& design cabinet condition monitoring system.
			\item Manage 1800+ follower Twitter account, respond to inquiries, and educate all ages at outreach events \& conferences.
		\end{itemize}
	
		{\bf DeSantis DREAM Lab }{Research Assistant} \hfill {\em 2016 - 2018} \\
		{Vanderbilt University} \hfill {Nashville, TN} 
		\begin{itemize}
			\itemsep -0.5em \vspace{-0.5em} \small
			\item Maintained lab schedules, supplies, and spaces while collecting 500+ microwear \& 200+ stable isotope data points.
			\item Trained 8+ users in confocal microscope use, stable isotope sample collection/preparation, data analysis, and scientific communication.
			\item Tested validity of dental microwear texture as well as stable carbon \& oxygen isotope paleo-environment proxies in Australian brushtail possums.
		\end{itemize}
		
		{\bf Museum of Natural and Cultural History }{Paleontology Collection Assistant} \hfill {\em 2014 - 2016} \\
		{University of Oregon} \hfill {Eugene, OR} 
		\begin{itemize}
			\itemsep -0.5em \vspace{-0.5em} \small
			\item Geo-referenced specimens and produced multi-institution training materials for NSF-funded EPICC TCN grant (\#1503065, \$606K).
			\item Manage specimen inventory and process incoming \& outgoing specimens.
		\end{itemize}
		
		{\bf UO Vertebrate Paleontology Lab }{Undergraduate Research Assistant} \hfill {\em 2013 - 2016} \\
		{University of Oregon} \hfill {Eugene, OR} 
		\begin{itemize}
			\itemsep -0.5em \vspace{-0.5em} \small
			\item Reviewed primary literature to construct database of modern primate diets.
			\item Used R \& ArcGIS (a geographic information system) to analyze \& visualize data e.g., Cenozoic fossil diversity.
			\item Identified multiple Miocene Sciuridae species to prepare taxonomic reports including tentative description of a new species, {\em Miospermophilus paulinaensis}.
		\end{itemize}

		{\bf Special Collections and University Archives }{Reading Room Assistant} \hfill {\em 2013 - 2014} \\
		{University of Oregon} \hfill {Eugene, OR}	
		\begin{itemize}
			\itemsep -0.5em \vspace{-0.5em} \small
			\item Identified \& retrieved physical \& digitized materials for researchers.
			\item Processed incoming collections, collected metadata, and digitized materials.
		\end{itemize}	
	\end{rSection}

	\begin{rSection}{Projects}
		\href{https://github.com/Eva-Biedron/EVAbase}{\bf EVAbase: an SQLite database for my mineral and fossil collection} \hfill {\em Ongoing}
		\\
		\\
		\\
	\end{rSection}

\begin{rSection}{Skills}	
	\begin{tabular}{ @{} >{\bfseries}l @{\hspace{4ex}} l }	
		Analytical & Predictive statistics, Time series analysis,  Relational databases, Agent-based modeling \\
		Collections & Accessioning, Cataloging \& curation, Specimen access \& loans, Conservation, Digitization \\
		Earth Science & Mineralogy \& Petrology, Hydrology, Sedimetology \& Stratigraphy, Geological mapping\\
		Software/Tools & Specify, MCZbase/Arctos, OpenRefine, MS Office, Adobe Photoshop \& Lightroom \\
		Coding/Tools & Python (pandas, NumPy, scikit-learn, \& seaborn), R, SQL, SQLiteStudio,  \LaTeX, Matlab \\
		Business & Project/process management, Agile methods, negotiation, stakeholder analysis \\
		Communication & Scientific communication, public speaking, social media management, grant-writing \\
		
	\end{tabular}		
\end{rSection}

	\begin{rSection}{Publications \& Products}
		{\bf Biedron, E.M.} (2018) \href{https://etd.library.vanderbilt.edu/etd-07202018-135518}{A multiproxy approach to tracking aridity across Australian landscapes using brushtail possums (Marsupialia: Phalangeridae: Trichosurus)} Masters of Science Thesis, Vanderbilt University, Nashville, TN.
			
		DeSantis, L.R.G., Alexander, J., {\bf Biedron, E.M.}, {\em et al.} (2018) \href{10.1371/journal.pone.0201962}{\em Effects of climate on dental mesowear of extant koalas...} PLOS ONE 13(8): 1-15.
		
		{\bf Biedron, E.M.} (2018) \href{https://scholarsbank.uoregon.edu/xmlui/handle/1794/20264}{The Sciuridae (Rodentia: Mammalia) of Cave Basin (Oregon), A New Middle Miocene Microfossil Locality} Honors Thesis, University of Oregon, Eugene, OR.
		
		{\bf Biedron, E.M.} \& Famosos, N.A. (2016) \href{https://epicc.berkeley.edu/wp-content/uploads/2015/11/UsingGeoLocateforCollaborativeGeoreferencing\_2016.pdf}{\em Using GEOLocate for Collaborative Georeferencing.} EPICC TCN Protocols, University of California Berkeley, Berkeley, CA.
	\end{rSection}
	
		\begin{rSection}{Grants}
		{\bf National Science Foundation }{Graduate Research Fund Proposal, Semi-Finalist} \hfill {\em 2017} \\
		{\bf Geological Society of America }{Graduate Research Grant, \$1740.00} \hfill {\em 2017}
	\end{rSection}	
	
		\begin{rSection}{Leadership}
		{\bf Association of Women Geoscientists }{Vanderbilt University Chapter. Treasurer, founding member} \hfill {\em 2017} \\
		{\bf Argentine Tango Club }{University of Oregon. Vice President} \hfill {\em 2016}
	\end{rSection}
	
	\begin{rSection}{Teaching}
		{\bf Earth and Environmental Sciences }{Teaching Assistant} \hfill {\em 2016, 2018} \\
		{Vanderbilt University, EES 1510 Dynamic Earth, EES 3220 Life Through Time} \hfill {Nashville, TN}
		
		{\bf Scientific Literacy Program }{Undergraduate Scholar} \hfill {\em 2014, 2016} \\
		{University of Oregon, GEOL 103 Evolving Earth} \hfill {Eugene, OR}
	\end{rSection}
	
	\begin{rSection}{Workshops}
		{\bf Inclusive Scientific Communication} Society of Vertebrate Paleontology \hfill {\em 2021} \\
		{\bf Introduction to Biodiversity Specimen Digitization} iDigBio \hfill {\em 2021} \\
		{\bf Data Carpentry} University of California Museum of Paleontology \hfill {\em 2016}
	\end{rSection}

	\begin{rSection}{Conference Presentations}
		\small
		Byrd, C., Guerrero, J., Spind, C. {\bf Biedron, E.M.} 2023. Assoc. of Materials \& Methods in Paleontology. Accepted. \\
		Byrd, C., {\bf Biedron, E.M.}, Green, C. \& J. Woodward. 2021. Society of Vertebrate Paleontology. Online. \\
		{\bf Biedron, E.M.} 2021. Society for the Preservation of Natural History Collections. Online. \\
		{\bf Biedron, E.M.} \& DeSantis, L.R.G. 2018. Society of Vertebrate Paleontology. Albuquerque, NM. \\		
		{\bf Biedron, E.M.}, \& DeSantis, L.R.G. 2018. Geological Society of America, Southeastern section. Knoxville, TN. \\
		DeSantis, L.R.G., Coulson, G., {\bf Biedron, E.M.}, {\em et al.} 2018. Geological Society of America, Southeastern section. 	Knoxville, TN. \\	
		{\bf Biedron, E.M.} \& L.R.G. DeSantis. 2017. Geological Society of America. Seattle, WA. \\	
		Davis, E.B., Hopkins, S.S.B., Famoso, N.A., {\bf Biedron, E.M.}, {\em et al.} 2017. Society of Vertebrate Paleontology. Calgary, AB, Canada. \\	
		McLaughlin, W., {\bf Biedron, E.M.}, Davis, E.B., {\em et al.} 2016. Geological Society of America. Denver, CO. \\
		{\bf Biedron, E.M.} \& Hopkins, S.S.B. 2016. Society of Vertebrate Paleontology. Salt Lake City, UT. \\	
		Hopkins, S.S.B., Davis, E.B., Theodor, J.M., McLaughlin, W., Reuter, D., Perdue, G., Oberg, D., {\bf Biedron, E.M.}, \& Walters, K. 2016. Society of Vertebrate Paleontology. Salt Lake City, UT. \\	
		{\bf Biedron, E.M.} \& Hopkins, S.S.B. 2016. University of Oregon Undergraduate Symposium. Eugene, OR. \\	
		{\bf Biedron, E.M.} \& Hopkins, S.S.B. 2016. EVO-WIBO. Port Townsend, WA. \\		
		{\bf Biedron, E.M.} \& Hopkins, S.S.B. 2015. Society of Vertebrate Paleontology. Dallas, TX. \\	
		{\bf Biedron, E.M.} \& Hopkins, S.S.B. 2014. Geological Society of America. Vancouver, BC, Canada. \\		
		{\bf Biedron, E.M.} \& Hopkins, S.S.B. 2014. University of Oregon Undergraduate Symposium. Eugene, OR. \\		
		{\bf Biedron, E.M.} \& Hopkins, S.S.B., \& McLaughlin, W. 2013. Society of Vertebrate Paleontology. Los Angeles, CA. \\
	\end{rSection}	

	\begin{rSection}{References}
		\normalsize Jessica Cundiff \\
		\small Curatorial Associate, Invertebrate Paleontology, Museum of Comparative Zoology \\
		
		\normalsize Scott Johnston \\
		\small Preparator \& Technician, Vertebrate Paleontology, Museum of Comparative Zoology \\
		
		Contact information available upon request.
	\end{rSection}	
\end{document}
